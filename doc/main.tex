\documentclass{article}
\usepackage{graphicx} % Required for inserting images
\usepackage{svg} % Required for inserting svg files
\usepackage{float} % Required for placing images in a precise location
\usepackage[a4paper, margin=2cm]{geometry}

\setlength{\parindent}{0cm} % Remove indentation of paragraph

\title{Carpenter}
\author{Federico Cipressi}
\date{December 2023}

\begin{document}

\maketitle

\section{Table}
Let's consider one side of the table and along that side we must arrange the chairs.

Let's call $t$ the length of the table, $c$ the length of the chair and $d$ the distance between the chairs.

\begin{figure}[H]
    \centering
    \includesvg{images/carpenter_1.svg}
\end{figure}

Let's call $n$ the number of chairs to arrange.

\begin{figure}[H]
    \centering
    \includesvg{images/carpenter_2.svg}
\end{figure}

$$n \cdot c + (n -1)d = t$$
$$nc + nd - d = t$$
$$n(c + d) = t + d$$
$$n = \frac{t + d}{c + d}$$
We consider $n$ as the integer division and let's call $r$ the reminder of $\frac{t + d}{c + d}$.

Now let's find the coordinates, considering only the abscissas.

\begin{figure}[H]
    \centering
    \includesvg{images/carpenter_3.svg}
\end{figure}

$O$ is the origin of axis, $T$ the center of the table, $A$ the center of the first chair and $B$ the center of the second chair.
$$A.x = O.x + \frac{r}{2} + \frac{c}{2}$$
$$B.x = A.x + c + d$$
If $O.x = T.x - \frac{t}{2}$, then
$$A.x = T.x - \frac{t}{2} + \frac{r}{2} + \frac{c}{2}$$
that is the formula used in the program.

\section{Chair}
The position of the backrest of the chair depends on its orientation:

\begin{figure}[H]
    \centering
    \includesvg[width=\textwidth]{images/carpenter_4.svg}
\end{figure}

Let's find the coordinates of the backrest. We consider the case of upward orientation, the other cases are analogous.

\begin{figure}[H]
    \centering
    \includesvg{images/carpenter_5.svg}
\end{figure}

$O$ is the origin of axis, $B$ and $b$ are the position and width of the backrest respectively; $C$ the center of the chair, $w$ and $l$ are the width and length of the chair respectively.
$$B.x = O.x$$
$$B.y = O.y + w - b$$
Knowing that $O.x = C.x - \frac{l}{2}$ and $O.y = C.y - \frac{w}{2}$,
$$B.x = C.x - \frac{l}{2}$$
$$B.y = C.y - \frac{w}{2} + w - b$$
and we get the formula used in the program
$$B.x = C.x - \frac{l}{2}$$
$$B.y = C.y + \frac{w}{2} - b$$
\end{document}
